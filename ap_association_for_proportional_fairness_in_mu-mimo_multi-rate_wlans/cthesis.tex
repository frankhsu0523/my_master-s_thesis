%!TEX encoding = UTF-8 Unicode 
\documentclass[a4paper,12pt]{report}
\usepackage{CJKutf8}
\usepackage{graphicx}
\usepackage[sf]{titlesec}

\topmargin 0cm \oddsidemargin 0.5in \headsep 1.5cm \headheight 0.5cm
\textheight 21.5cm \textwidth 15.2cm \marginparwidth 0cm
\marginparsep 0cm
\floatsep 0.8cm     %頁之上端與下端圖表與圖表之間的間距
\textfloatsep 0.8cm %頁之上端與下端圖表與文字之間的間距
\intextsep 1.0cm    %指定為 h 的圖表與文字之間的間距
%\jot 0cm       %Array 數式間的間距
%\abovedisplayskip 1.2cm    %數式的上端與其上面文章間的間距
%\belowdisplayskip 1.2cm    %數式的下端與其下面文章間的間距
%\abovedisplayshortskip 1.2cm   %數式的上端與其上面文章間的間距
%\belowdisplayshortskip 1.2cm   %數式的下端與其下面文章間的間距

\newtheorem{alg}{Algorithm}
\newtheorem{lemma}{Lemma}
\newtheorem{theorem}{Theorem}
\newtheorem{proof}{Proof}
\newtheorem{definition}{Definition}
\newtheorem{example}{Example}
\newtheorem{statement}{Statement}
\newtheorem{corollary}{Corollary}

\newcommand{\doublespace}
        {\addtolength{\baselineskip}{.6\baselineskip}}

\begin{document}

\begin{CJK}{UTF8}{bkai}

\titleformat{\chapter}[display]{\Huge}
            {第\ \thechapter\ 章}{0.2cm}{}
\renewcommand{\contentsname}{目錄}

%\setlength{\baselineskip}{22pt}

\begin{titlepage}
\begin{center}
{\fontsize{28}{28}國~  立~  清~  華~  大~  學}\\

\vspace{1.5cm}
\Huge{\underline{碩 士 論 文}}\\
\vspace{0.8cm}
%\Large{(草稿)}\\
\vspace{2.7cm} \Large{在多用戶多輸入多輸出的無線區域網路環境下,達成比例公平分配條件的無線網路基地台連接配對問題
}\\
\vspace{0.5cm} \Large{\bf AP Association for Proportional Fairness in MU-MIMO Multi-rate WLANs\\}

\vspace{3cm}
\parbox[t]{14cm}{
%實際寬度自行再調整
\doublespace{
{\bf 系  別}:\underline{ 資~訊~工~程~學~系  ~~}{\bf 組別}:\underline{   ~~}\\
{\bf 學號姓名}:\underline{ 102062650 陳~鈺~書(Yu-Shu~Chen)~~}\\
{\bf 指導教授}:\underline{ 蔡~明~哲 博士~~(Dr.~Ming~Jer~Tsai)}\\
}}

\vspace{1cm}
中華民國一百零四年七月\\
\end{center}
\end{titlepage}

\cleardoublepage \pagenumbering{roman} \doublespace

\chapter*{中文摘要}
\addcontentsline{toc}{chapter}{中文摘要}

\indent
在無線區域網路裡無線網路基地台連接配對問題的演算法設計已經受到許多人的關注,這是因為分配給使用者的頻寬會隨著演算法的不同而有差異。為了要讓每個使用者所被分配到的頻寬能夠符合他的最低需求,在無線網路基地台連接配對問題需要去考慮公平性。已經存在的演算法只有考慮在每個無線網路基地台在某個時間點只能跟一個使用者溝通的環境下,去達到頻寬分配公平性最佳化的狀態。然後現在藉著使用波束成型技術可以讓有多根天線的無線網路基地台能夠同時跟多位使用者傳遞資料。

在這篇論文中,我們探討在多用戶多輸入多輸出的無線區域網路環境下,達成比例公平分配條件的無線網路基地台連接配對問題來使得分配給使用者頻寬的效用函數的和最大化。就我們所知,我們是第一個來探討這個問題的。在這篇論文中,我們會證明這是一個NP-hard 的問題並且提出一個近似演算法使得它的答案會不小於 $OPT - |U|\log(|A|)$,在這裡 $U$ 是使用者的集合,$A$ 是無線網路基地台的集合。

從模擬的結果可以看出我們所提出的演算法對於分配給使用者頻寬的效用函數的和有良好的表現。\\
~\\
\noindent
\\

\cleardoublepage



\end{CJK}

\end{document}
