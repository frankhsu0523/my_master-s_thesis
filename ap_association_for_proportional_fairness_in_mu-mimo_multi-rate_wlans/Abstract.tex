% Abstract
Design of AP association algorithms has received considerable attentions in wireless local area networks (WLANs) because the  bandwidth allocated to users is decided by the AP association algorithms designed. In order to make the allocated bandwidth of each user satisfy the lower bound of the requirement of each user, AP association needs to be considered about fairness. The existing algorithms concerning the total bandwidth optimization problem under fairness constraints only allow each AP to be able to communicate with at most one user in a time slot. However, today, by the beamforming technique makes it possible that each AP with multiple antennas can transmit data to multiple users simultaneously.
		
In this thesis, we study the AP association for proportional fairness, termed AAPFM, to maximize the sum of the utility function of the bandwidth allocated to each user in MU-MIMO WLANs. To the best of our knowledge, we are first to study the AP association problem for proportional fairness in MU-MIMO WLANs. In this thesis, we show the AAPFM problem is NP-hard and propose an approximation algorithm that the solution is greater than $OPT - |U|\log{|A|}$ where $U$ is the set of users and $A$ is the set of APs.
		
Simulations for evaluations show that the proposed AP association algorithm has good performance in terms of the sum of the utility function of the bandwidth allocated to each user.